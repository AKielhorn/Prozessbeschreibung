% !TEX TS-program = ConTeXt (LuaTeX 1.0.9)
% Copyright 2019 Axel Kielhorn
% Lizenz: CC-BY-SA 4.0 Unported http://creativecommons.org/licenses/by-sa/4.0/deed.de

% Aufruf mit
% context --environment=prozess-style.tex prozess.xml

\startluacode
settings = {}
docstruktur = {}
doclistfile = "doclist.xml"

local striplines = utilities.strings.striplines
local xmltext = xml.text

doc = xml.load(doclistfile, settings)

for v in xml.collected(doc,"/doclist/psdoc/") do
 local docnr = striplines(xmltext(v,"/docnr"))
 local docname = striplines(xmltext(v,"/docname"))
 -- es kann mehrere DOCAN geben!
 local docan = striplines(xmltext(v,"/docan"))
 local docverantwortlich = striplines(xmltext(v,"/docverantwortlich"))
 docstruktur[docnr]={
   docname = docname,
   docan = docan,
   docverantwortlich = docverantwortlich
 }
end
\stopluacode

\startluacode
local striplines = utilities.strings.striplines
local xmltext = xml.text
function xml.finalizers.tex.PsOutdoc(e)
    local docnr = striplines(xml.text(e[1],"//docnr"))
    local c = docstruktur[docnr]
    context ("\\inmargin{" .. c["docan"] .. " $\\Leftarrow$}" .. "{\\bf " .. docnr .. "} " .. c["docname"])
end
\stopluacode

\startluacode
local striplines = utilities.strings.striplines
local xmltext = xml.text
function xml.finalizers.tex.PsIndoc(e)
    local docnr = striplines(xml.text(e[1],"//docnr"))
    local c = docstruktur[docnr]
    context ("\\inmargin{" .. c["docverantwortlich"] .. " $\\Rightarrow$}" .. "{\\bf " .. docnr .. "} " .. c["docname"])
end
\stopluacode

\language[de]
\setbreakpoints[compound]

\setuphyphenation[method=expanded]

\setuppapersize [A4,landscape][A4,landscape]
\setuplayout    [width=middle,  backspace=2in, cutspace=1in, leftmargin=1.5in,
                 height=middle, topspace=0.75in, bottomspace=0.75in]

\setuppagenumbering[location={footer,right}]

\setupheader[text][start]
\setupfooter[text][start]
%\setupheadertexts[][\date]

 \setupheadertexts[title][\date]
 \setuphead[title]
   [placehead=empty,
    before=,
    after=,
    page=,]

\startxmlsetups xml:prozessb
    \xmlsetsetup{#1}{*}{-}
    \xmlsetsetup{#1}{prozess|pschritt|psindoc|psoutdoc|aschritt|eschritt}{xml:*}
\stopxmlsetups

\xmlregistersetup{xml:prozessb}

\startxmlsetups xml:prozess
  \title{\xmltext{#1}{/pnr} \xmltext{#1}{/pname}}
%  \setupheadertexts[\bf\xmltext{#1}{/pnr} \xmltext{#1}{/pname}][\date]
  {\bTABLE
    \setupTABLE[frame=off]
    \setupTABLE[offset=1mm]
    \setupTABLE[c][1][align=right, width=14cm]
    \setupTABLE[c][2][align=left,  width=6cm]
    \setupTABLE[c][3][align=right, width=4cm]
    \bTR \bTD \bfb\xmltext{#1}{/pnr}   \eTD \bTD Version:        \eTD \bTD \bf\xmltext{#1}{/pversion}\eTD \eTR
    \bTR \bTD \bfb\xmltext{#1}{/pname} \eTD \bTD Verantwortlich: \eTD \bTD \bf\xmltext{#1}{/pverantwortlich}\eTD\eTR
  \eTABLE}
  \blank
  \xmlflush{#1}
\stopxmlsetups

\startxmlsetups xml:pschritt
    \inmargin{\bf \xmltext{#1}{../pnr}.\xmltext{#1}{/psnr}}
    {\bf \xmldoifelse{#1}{/pslang}{\xmltext{#1}{/pslang}}{\xmltext{#1}{/pskurz}}}
    \blank
    \xmlflush{#1}
    \blank
\stopxmlsetups

\startxmlsetups xml:psindoc
    \xmlfilter{#1}{./PsIndoc()}
%    \inmargin{\xmltext{#1}{/docverantwortlich} $\Rightarrow$} {\bf \xmltext{#1}{/docnr}} \xmltext{#1}{/docname}
    \blank
\stopxmlsetups

\startxmlsetups xml:psoutdoc
    \xmlfilter{#1}{./PsOutdoc()}
%    \inmargin{\xmlconcat{#1}{/docan}{, } $\Leftarrow$}  {\bf \xmltext{#1}{/docnr}} \xmltext{#1}{/docname}
    \blank
\stopxmlsetups

%\startxmlsetups xml:docan
%   \xmlflush{#1}
% \stopxmlsetups

\startxmlsetups xml:aschritt
    \color[darkgreen]{
    \inmargin{\color[darkgreen]{\xmltext{#1}{/asverantwortlich}}}
    {\xmldoifelse{#1}{/aslang}{\xmltext{#1}{/aslang}}{\xmltext{#1}{/askurz}}}
    \xmlflush{#1}
    }
    \blank
\stopxmlsetups

\startxmlsetups xml:eschritt
    \color[red]{
    {\xmldoifelse{#1}{/eslang}{\xmltext{#1}{/eslang}}{\xmltext{#1}{/eskurz}}}
    \xmlflush{#1}}
    \blank
\stopxmlsetups
