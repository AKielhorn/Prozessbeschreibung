% !TEX encoding = UTF-8 Unicode
% !TEX TS-program = pdflatex
% Copyright 2019 Axel Kielhorn
% Lizenz: CC-BY-SA 4.0 Unported http://creativecommons.org/licenses/by-sa/4.0/deed.de

%\documentclass[draft]{article}
\documentclass{article}

\usepackage[utf8]{inputenc}
\usepackage[ngerman]{babel}

\usepackage{tikz}
\usetikzlibrary{shapes,arrows,shapes.symbols}

\usepackage{tcolorbox}
\tcbuselibrary{skins}

\usepackage[active,tightpage]{preview}
\PreviewEnvironment{tikzpicture}
\setlength\PreviewBorder{5pt}%

\begin{document}
\pagestyle{empty}

\tikzstyle{line} = [draw, very thick, color=black!50, -latex']

\newtcolorbox{pdoku}[2][]{%
enhanced,
width=8em,colback=black!5,colframe=black!60,
nobeforeafter,
left=0mm, right=0mm,
boxsep = 2 mm,
height from = 4 em to 10cm,
%show bounding box,
halign title=center,
attach boxed title to top center={yshift=-\tcboxedtitleheight/2},
boxed title style={size=small,colback=black!60},
title={#2},
#1}

\newtcolorbox{pschritt}[2][]{%
enhanced,
width=8em,colback=black!20,colframe=black!75,
nobeforeafter,
%left skip=-1mm,
left=0mm, right=0mm,
boxsep = 2 mm,
height from = 4 em to 10cm,
%show bounding box,
halign title=center,
attach boxed title to top center={yshift=-\tcboxedtitleheight/2},
boxed title style={size=small,colback=black!75},
title={#2},
#1}

\newtcolorbox{action}[2][]{%
enhanced,
width=8em,colback=green!20,colframe=green!80,
nobeforeafter,
%left skip=-1mm,
left=0mm, right=0mm,
boxsep = 2 mm,
height from = 4 em to 10cm,
%show bounding box,
halign title=center,
attach boxed title to top center={yshift=-\tcboxedtitleheight/2},
boxed title style={size=small,colback=green!80},
title={#2},
#1}

\newtcolorbox{ereignis}[1][]{%
enhanced,
width=7em,colback=red!20,colframe=red!75,
nobeforeafter,
%left skip=-1mm,
left=0mm, right=0mm,
boxsep = 2 mm,
arc is angular,
arc=4mm,
height from = 4 em to 10 cm,
%show bounding box,
%halign title=center,
%attach boxed title to top center={yshift=-\tcboxedtitleheight/2},
%boxed title style={size=small,colback=red!75},
%title={#2},
#1}

\begin{tikzpicture}[scale=2, node distance = 4cm, auto]
% Prozessschritte
    \node                (p1)  {\begin{pschritt}{1.0.010}{Prozess beschreiben}\end{pschritt}};
    \node [below of=p1]  (p2)  {\begin{pschritt}{1.0.020}{Prozess visualisieren}\end{pschritt}};
    \node [below of=p2]  (p3)  {\begin{pschritt}{1.0.030}{Dokumenten- fluss erstellen}\end{pschritt}};
% PS1
    \node [right of=p1]  (a1) {\begin{action}{EDV}{DTD erstellen}\end{action}};
    \node [right of=a1]  (e1) {\begin{ereignis}{DTD validiert}\end{ereignis}};
    \node [right of=e1]  (a2) {\begin{action}{MA}{XML erstellen}\end{action}};
    \node [right of=a2]  (e2) {\begin{ereignis}{XML validiert}\end{ereignis}};
% PS2
    \node [right of=p2]  (a3) {\begin{action}{EDV}{Tabellarische Version erstelllen}\end{action}};
    \node [right of=a3]  (e3) {\begin{ereignis}{Tabellarische Form auf Webserver}\end{ereignis}};
    \node [right of=e3]  (a4) {\begin{action}{EDV}{Graphische Version erstellen}\end{action}};
    \node [right of=a4]  (e4) {\begin{ereignis}{Graphische Form auf Webserver}\end{ereignis}};
% PS3
    \node [right of=p3]  (stop) {\begin{ereignis}{Stop}\end{ereignis}};
% Doku
    \node [above left of=p1]   (dokin1)   {\begin{pdoku}{LA}{DTD.html}\end{pdoku}}; 
    \node [left of=dokin1]     (dokin2)   {\begin{pdoku}{LA}{Prozess- beschreibung.html}\end{pdoku}}; 
    \node [below left of=p2]   (dokout21) {\begin{pdoku}{$\Rightarrow$ Web-Server}{Prozess-t.pdf}\end{pdoku}}; 
    \node [left of=dokout21]   (dokout22) {\begin{pdoku}{$\Rightarrow$ Web-Server}{Prozess-g.pdf}\end{pdoku}}; 
    \node [below left of=p3]   (dokout3) {\begin{pdoku}{$\Rightarrow$  Web-Server}{Doku-fl.xml}\end{pdoku}}; 
    \node [below left of=p1]   (dokout1) {\begin{pdoku}{$\Rightarrow$ PL}{Prozess.xml}\end{pdoku}}; 
% Verbindungen
% PS1
    \path [line] (p1) -- (a1);
    \path [line] (a1) -- (e1);
    \path [line] (e1) -- (a2);
    \path [line] (a2) -- (e2);
    \path [line] (e2) -- +(0,-1)  -| (p2);
    \path [line] (dokin1) |- (p1.170);
    \path [line] (dokin2) |- (p1.170);
    \path [line] (p1.200) -| (dokout1);
% PS2
    \path [line] (p2) -- (a3);
    \path [line] (a3) -- (e3);
    \path [line] (e3) -- (a4);
    \path [line] (a4) -- (e4);
    \path [line] (e4) -- +(0,-1)  -| (p3);
    \path [line] (p2.200) -| (dokout21);
    \path [line] (p2.200) -| (dokout22);
% PS3
    \path [line] (p3) -- (stop);
    \path [line] (p3.200) -| (dokout3);
\end{tikzpicture}


\end{document}
